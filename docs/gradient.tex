\documentclass[12pt]{article}
\usepackage{amsmath}
\usepackage{txfonts}

%- Abbreviations.
\newcommand{\psf}{\mathbf{H}}
\newcommand{\G}{\mathbf{G}}
\newcommand{\hhat}{\mathbf{\hat h}}
\newcommand{\x}{\mathbf{x}}
\newcommand{\xhat}{\mathbf{\hat x}}
\newcommand{\y}{\mathbf{y}}
\newcommand{\rvect}{\mathbf{r}}
\newcommand{\W}{\mathbf{W}}
\newcommand{\A}{\mathbf{A}}
\newcommand{\avec}{\mathbf{a}}
\newcommand{\F}{\mathbf{F}}


\newcommand{\ddx}{\frac{\partial}{\partial \mathbf{x}}}
\newcommand{\ddxi}{\frac{\partial}{\partial \mathbf{x}_{i}}}

\newcommand{\Nsn}{\textbf{XXX}}
\newcommand{\Nspec}{\textbf{YYY}}


\newcommand{\rem}[1]{\textbf{[\textsl{#1}]}}
\newcommand{\szf}[1]{\textbf{\textsl{SZF: #1}}}

% -----------------------------------------------------------------------------
% Document

\begin{document}

\title{CubeFit $\chi^2$ gradient calculation}
\author{S.~Bongard, Kyle~Barbary, Clare~Saunders}
\maketitle
\begin{abstract}
In CubeFit, one of the fitting steps is to fit the galaxy model to the
data while holding other parameters (such as positions) fixed. The
galaxy model has on order $\sim$ 1 million parameters. The ability to
analytically calculate the gradient (a vector of length ~$\sim$ 1
million) on the fit $\chi^2$ is essential in order to carry out the fit
efficiently. This note explains the derivation of the gradient.
\end{abstract}


\section{$\chi^2$ of Galaxy Fit}

The $\chi^2$ in the galaxy fit is given by

\begin{equation}
\chi^2 = (\y - \G\x)\W(\y - \G\x)
\end{equation}

where $\y$ is the data, $\x$ is the model parameters, and $\G$ is the operator
that translates the model into the data frame. $\G$ includes the effect of PSF
convolution, shift operation and cropping. (Note that we are supposing that
we can represent all these operations as a matrix $\G$, but we haven't shown
this.) Remember that this is all in real space at this point. Carrying on,

\begin{eqnarray*}
  \label{eq:5}
  \ddx \chi^{2} & = & \ddx \left( (\y - \G\x)^{T} \W (\y - \G\x) \right)\\
                & = & \ddx \left( (\G \x)^{T} \W \y \right)
                    + \ddx \left( \y^{T} \W \G \x \right)
                    - \ddx \left( (\G \x)^{T} \W \G \x \right) \\
                & = & \ddx \left( \x^{T} \G^{T} \W \y \right)
                    + \ddx \left( (\G^{T} \W \y)^{T} \x \right)
                    - \ddx \left( \x^{T} \G^{T} \W \G \x \right) \\
\end{eqnarray*}

where in the last line we have used the property that
$(\mathbf{A}\mathbf{B})^T = \mathbf{B}^T\mathbf{A}^T$ and assumed that
$\W = \W^T$ (the weight matrix is symmetric).

We note that $(\G^{T} \W \y)^{T}$ and $\G^{T} \W \y$ are vectors ($\y$ is a
vector). The first two terms of the equation are thus of the form $\ddx \x^T \avec$
and $\ddx \avec^T \x$. These are both simply equal to $\avec$ as $\x^T \avec$ and
$\avec^T \x$ are dot products between the constant vector $\avec$ and $\x$, so the
partial derivative with respect to $\x$ is just $\avec$. Making this substitution,
we have

\begin{equation}
\ddx \chi^{2} = 2 \G^T \W \y - \ddx \left( \x^{T} \G^{T} \W \G \x \right)
\end{equation}

The second term has the form $\ddx \x^T\A\x$, with
$\A = \G^T \W \G$. It helps to write this out:

\begin{eqnarray*}
  \ddxi \x^{T}\A\x & = & \ddxi \left( \sum_{j} x_{j} \sum_l A_{j,l} x_{l}  \right) \\
                   & = & \sum_l A_{i,l} x_{l} + \sum_j A_{j,i} x_{j} \\
                   & = & \left( \A \x + \A^{T} \x␇ \right)_{i}
\end{eqnarray*}

so

\begin{equation}
  \ddx \x^T\A\x = (\A + \A^T) \x .
\end{equation}

Substituting this in (with $\A = \G^T \W \G$), we finally get:

\begin{eqnarray*}
\ddx \chi^{2} & = & 2 \G^T \W \y - \left( \G^T \W \G \x + (\G^T \W \G)^T \x \right) \\
              & = & 2 \G^T \W \y - 2 \G^T \W \G \x \\
              & = & 2 \G^T \W ( \y - \G \x ) \\
\end{eqnarray*}

But what is the matrix $\G$? For that we need to digress about Fourier transforms
as matricies...

\section{Matrix representation of the discrete Fourier transform}

The discrete Fourier transform of a 1-d array $x_n$ is

\begin{equation}
X_k = \sum_{n=0}^{N-1} x_n e^{-i 2 \pi k n / N}
\end{equation}

We can write this in matrix form,

\begin{equation}
\mathbf{X} = \left( \begin{array}{cccc}
   e^{-i 2\pi 0 \cdot 0 / N} & e^{-i 2\pi 0\cdot 1 / N} & e^{-i 2\pi 0 \cdot 2 / N} & \cdots \\
   e^{-i 2\pi 1 \cdot 0 / N} & e^{-i 2\pi 1 \cdot1 / N} & e^{-i 2\pi 1 \cdot 2 / N} & \cdots \\
   e^{-i 2\pi 2 \cdot 0 / N} & e^{-i 2\pi 2 \cdot 1 / N} & e^{-i 2\pi 2 \cdot 2 / N} & \cdots \\
   \vdots & \vdots & \vdots & \ddots
   \end{array} \right)
   \left( \begin{array}{c} x_0 \\ x_1 \\ x_2 \\ \vdots \end{array} \right)
\end{equation}

For a shorthand, we will call the square matrix $\mathbf{F}$

\begin{equation}
\mathbf{X} = \mathbf{F} \mathbf{x}
\end{equation}

Similarly, the inverse transform can be written

\begin{equation}
x_n = \frac{1}{N} \sum_{k=0}^{N-1} X_k e^{i 2 \pi k n / N}
\end{equation}

or

\begin{equation}
   \mathbf{x} = \frac{1}{N} \left( \begin{array}{cccc}
   e^{i 2\pi 0 \cdot 0 / N} & e^{i 2\pi 1 \cdot 0 / N} & e^{i 2\pi 2 \cdot 0 / N} & \cdots \\
   e^{i 2\pi 0 \cdot 1 / N} & e^{i 2\pi 1 \cdot 1 / N} & e^{i 2\pi 2 \cdot 1 / N} & \cdots \\
   e^{i 2\pi 0 \cdot 2 / N} & e^{i 2\pi 1 \cdot 2 / N} & e^{i 2\pi 2 \cdot 2 / N} & \cdots \\
   \vdots & \vdots & \vdots & \ddots
   \end{array} \right)
   \left( \begin{array}{c} X_0 \\ X_1 \\ X_2 \\ \vdots \end{array} \right)
\end{equation}

or as shorthand,

\begin{equation}
\mathbf{x} = \mathbf{F}^{-1} \mathbf{X}
\end{equation}

Now, note that the matrix $\mathbf{F}$ has the property

\begin{equation}
\mathbf{F}^{-1} = \frac{1}{N} \bar{\mathbf{F}}
\end{equation}

where $\bar{\mathbf{F}}$ is the element-wise complex
conjugate of $\mathbf{F}$.

\section{Convolution in Fourier space}

Convolution in real space is equivalent to element-wise multiplication (the
Hadamard product, denoted by $\circ$) in Fourier space. Thus, if $\avec$
represents the PSF in real space, the convolution of the galaxy in Fourier
space is $(\F\A) \circ \F\x$




\section{Fourier transform form}

The discrete Fourier transform of a vector $\x$ is:

\begin{equation}
  \label{eq:7}
  F(\x)_{k} = \sum_{j}^{N} x_{j} e^{- \frac{2\pi k j}{N}} = (\F \x)_{k}
\end{equation}

And the inverse discrete Fourier transform of vector $\xhat$ is
\begin{equation}
  \label{eq:8}
  F(\xhat)_{j} = \frac{1}{N} \sum_{k}^{N} \hat x_{k} e^{ \frac{2\pi j k}{N}} =
  (\F^{-1} \xhat)_{j}
\end{equation}

We chose to use the $1/N$ factor in the inverse transform because it makes the
normalization easier to carry accros calculations.

Since we can switch $j$ and $k$ in the exponentials above, we see that the
discrete direct and inverse Fourier transform in matricial form are related by
the formulae:
\begin{eqnarray*}
  \label{eq:9}
  \F^{-1} & = &\frac{1}{N} \F^{*T} \\
  \F^{-1*T} & = &\frac{1}{N} \F \\
\end{eqnarray*}
where $^{*}$ denotes the complex conjugation.


Let us now introduce the $diag()$ operator, which transforms a vector into a
diagonal matrix, the diagonal being the vector it operates upon: if $\A =
diag(\avec)$, then $\A_{i,j} = \delta_{i,j} a_{i}$. The vector obtained by an
element to element multiplication between $\x$ and $\avec$ can thus be writen
$diag(\avec) \x$.

In matrix form, the Fourier transform of the PSF convolution of vector $\x$ thus
writes as the element to element product of $\hhat = \F(\psf)$ and $\xhat =
\F(\x)$.

The operator \emph{convolution by the PSF} thus writes in matricial form:
\begin{equation}
  \label{eq:11}
  \psf = \F^{-1} diag(\hhat) \F
\end{equation}

From the equation of the matricial derivative of the $\chi^{2}$ derived in the
previous section, we also need to now how to write $\psf^{T} = \F^{T}
diag(\hhat)^{T} \F^{-1T}$ in a form that has the Fourier transforms in the same
order.

Since $\psf$ is a Real operator, we have $\psf = \psf^{*}$, and thus $\psf^{T} =
\psf^{*T}$.

Therefore, using the relations between direct and indirect Fourier transposes
derived above and the fact that the transpose of a diagonal matrix is the matrix
itself, we find:
\begin{eqnarray*}
  \label{eq:12}
  \psf^{T} & = & \F^{T} diag(\hhat)^{T} \F^{-1T}\\
           & = & (\F^{T} diag(\hhat)^{T} \F^{-1T})^{*}\\
           & = & \F^{T*} diag(\hhat)^{T*} \F^{-1T*}\\
           & = & N\F^{-1} diag(\hhat)^{*} \frac{1}{N}\F\\
           & = & \F^{-1} diag(\hhat)^{*} \F\\
\end{eqnarray*}

If we now write the residual $\rvect = \psf \x - \y$, we finally find that the
derivative of the $\chi^{2}$ writes:
\begin{equation}
  \label{eq:13}
  \ddx \chi^{2} = \F^{-1} diag(\hhat)^{*} \F(\W \mathbf{\rvect})
\end{equation}



% older stuff
\section{Appendix: Random extra note}

By writing explicitly $\partial (\G \x)_{j} / \partial x_{i}
= \partial/\partial x_{i} \left( \sum_{l} H_{j,l} x_{l}\right)$ and its transpose
it is easy to show the two following formulae. It is just important to make sure
to write the product in the exact same way depending on if $\x$ is on the left
or on the right of $\G$: $\left( \G \x \right)_{j} =  \sum_{l} H_{l,j} x_{l}$
and $\left( \x^{T} \G \right)_{j} =  \sum_{l} x_{l} H_{j,l}$

\begin{equation}
  \label{eq:3}
  \ddx \left( \x^{T} \G \right) = \ddx \left( \x^{T} \G \right)^{T} = \G^{T}
\end{equation}

\begin{equation}
  \label{eq:4}
  \ddx \left( \G \x \right) =  \ddx \left( \G \x \right)^{T} = \G
\end{equation}





\end{document}
